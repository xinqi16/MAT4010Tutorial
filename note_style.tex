% ========== 颜色定义 ==========
% 主文本颜色(柔和的深灰色)
\definecolor{maintext}{HTML}{2C3E50}        % 更深的蓝灰色
% 链接颜色(柔和的深蓝色)
\definecolor{linkblue}{HTML}{004085}        % 深蓝色
% 定理环境颜色
\definecolor{theoremcolor}{HTML}{2980B9}      % 深蓝色
\definecolor{definitioncolor}{HTML}{27AE60}   % 深绿色
\definecolor{propositioncolor}{HTML}{8E44AD}  % 深紫色
\definecolor{lemmacolor}{HTML}{D35400}        % 深橙色
\definecolor{corollarycolor}{HTML}{16A085}    % 深青绿色
\definecolor{examplecolor}{HTML}{E67E22}      % 深琥珀色
\definecolor{remarkcolor}{HTML}{7F8C8D}       % 深灰色
\definecolor{problemcolor}{HTML}{C0392B}       % 深红色

% 设置全局文本颜色
\color{maintext}

% 更新链接颜色使用深色柔和蓝色
\hypersetup{
    linkcolor=linkblue,
    citecolor=linkblue,
    urlcolor=linkblue,
    unicode=true
}

% ========== 定理环境定义 ==========
% 各环境独立计数,并添加颜色

\newtheoremstyle{theoremstyle}
  {3pt}   % 上方间距
  {3pt}   % 下方间距
  {\normalfont} % 正文字体
  {}      % 缩进
  {\bfseries\color{theoremcolor}} % 标题字体
  {.}     % 标题后标点
  {5pt plus 1pt minus 1pt} % 标题后间距
  {}      % 标题说明

\newtheoremstyle{definitionstyle}
  {3pt}
  {3pt}
  {\normalfont}
  {}
  {\bfseries\color{definitioncolor}}
  {.}
  {5pt plus 1pt minus 1pt}
  {}

\newtheoremstyle{propositionstyle}
  {3pt}
  {3pt}
  {\normalfont}
  {}
  {\bfseries\color{propositioncolor}}
  {.}
  {5pt plus 1pt minus 1pt}
  {}

\newtheoremstyle{lemmastyle}
  {3pt}
  {3pt}
  {\normalfont}
  {}
  {\bfseries\color{lemmacolor}}
  {.}
  {5pt plus 1pt minus 1pt}
  {}

\newtheoremstyle{corollarystyle}
  {3pt}
  {3pt}
  {\normalfont}
  {}
  {\bfseries\color{corollarycolor}}
  {.}
  {5pt plus 1pt minus 1pt}
  {}

\newtheoremstyle{examplestyle}
  {3pt}
  {3pt}
  {\normalfont}
  {}
  {\bfseries\color{examplecolor}}
  {.}
  {5pt plus 1pt minus 1pt}
  {}

\newtheoremstyle{remarkstyle}
  {3pt}
  {3pt}
  {\normalfont}
  {}
  {\itshape\color{remarkcolor}} % Remark通常用斜体
  {.}
  {5pt plus 1pt minus 1pt}
  {}

\newtheoremstyle{problemstyle}
  {3pt}
  {3pt}
  {\normalfont}
  {}
  {\bfseries\color{problemcolor}}
  {.}
  {5pt plus 1pt minus 1pt}
  {}

% 应用定理样式
\theoremstyle{theoremstyle}
\newtheorem{theorem}{Theorem}[section]

\theoremstyle{definitionstyle}
\newtheorem{definition}{Definition}[section]

\theoremstyle{propositionstyle}
\newtheorem{proposition}{Proposition}[section]

\theoremstyle{lemmastyle}
\newtheorem{lemma}{Lemma}[section]

\theoremstyle{corollarystyle}
\newtheorem{corollary}{Corollary}[section]

\theoremstyle{examplestyle}
\newtheorem{example}{Example}[section]

\theoremstyle{remarkstyle}
\newtheorem{remark}{Remark}[section]

\theoremstyle{problemstyle}
\newtheorem{problem}{Problem}[section]

% 可选:为证明环境添加样式
\usepackage{etoolbox}
\AtBeginEnvironment{proof}{\color{maintext}} % 确保证明文本也是主文本颜色